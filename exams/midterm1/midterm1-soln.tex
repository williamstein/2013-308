\documentclass[11pt]{article}
\usepackage{fullpage}
\usepackage{color}
\usepackage{amssymb}
\title{\color{blue}\bf Math 308: Midterm 1 Solutions}
\date{Wednesday, April 24, 2013}
\author{by William Stein}
\newcommand{\tf}[2]{\item {\bf {\color{blue}#1}}\hspace{1em} #2}
\begin{document}
\maketitle
{\noindent\bf\color{red} Instructions:}
\begin{itemize}
\item {\color{blue} Your Name: \underline{\hspace{20em}}   ID Number: \underline{\hspace{10em}}}
\item You may have one piece of paper on which you put notes, plus scratch paper.
\item You may use a calculator, though it will likely be of no use, since there is a table of reduced row echelon forms at the end of the exam.
\item In every question, assume that you are working over the real numbers (no complex numbers, no numbers modulo $p$).
\item Be careful.  If you can think of any possible way to double check your work do so.  Try to get a perfect score!
\item All questions are true or false and worth 4 points (for a total of 100).
\end{itemize}

{\noindent\bf \color{red} Questions (circle T or F):}
\begin{enumerate}

\tf{T}{There is a system of linear equations that has exactly one solution.}

\tf{F}{There is a system of linear equations that has exactly two solutions.}

\tf{T}{There is a system of linear equations that has infinitely many solutions.}

\tf{T}{The vector $(\pi,e)$ is a linear combination of the vectors $(1,2)$ and $(3,4)$ in $\mathbb{R}^2$.}

\tf{T}{The vector $(9,8,7)$ is a linear combination of the vectors $(1,2,3)$ and $(4,5,6)$.}

\tf{F}{The vectors $(9,10,11,12)$, $(1,2,3,4)$ and $(5,6,7,8)$ in $\mathbb{R}^4$ are linearly independent.}

\tf{F}{We have ${\rm span}((10,20),(20,10)) = {\rm span}((1,2))$.}

\tf{F}{We have ${\rm span}((1,2,3,4)) = {\rm span}((4,3,2,1))$.}

\tf{T}{We have ${\rm span}((1,2),(3,4)) = {\rm span}((0,1),(1,0))$.}

\tf{F}{Let $A=\left(\begin{array}{rrr}1 & 2 & 3 \\4 & 5 & 6 \\7 & 8 & 9\end{array}\right)$.  Then $\{ x : Ax=0 \} = {\rm span}((1,2,1)).$}

\tf{F}{Any two vectors in $\mathbb{R}^{4}$ are linearly independent.}

\tf{T}{Any four vectors in $\mathbb{R}^3$ are linearly dependent.}

\tf{F}{There are exactly three subspaces of $\mathbb{R}^2$.}

\tf{T}{If $A$ is a square $n\times n$ matrix, then $A + A^T$ is necessarily symmetric.}

\tf{T}{If $A$ and $B$ are $2\times 2$ matrices, then $(A+B)(B+A)^T$ is necessarily symmetric.}

\tf{T}{If $A$ is symmetric, then $A^T$ is necessarily also symmetric.}

\tf{F}{Suppose $A$ is an $n\times r$ matrix (i.e., has $n$ rows and $r$ columns), that $B$ is an $r\times m$ matrix, and that $C$ is an $m\times s$ matrix.  Then $ABC$ is defined and is an $s\times n$ matrix.}

\tf{T}{Suppose $A,B,C$ are all $3\times 3$ matrices. Then  $AB + (B + C) = C + (B + AB)$?}

\tf{F}{If $A+I=\left(\begin{array}{rr}
2 & 2 \\\\
2 & 5
\end{array}\right)$, then there is exactly one $x$ that such that $Ax = \left(\begin{array}{r}2013 \\\\2014\end{array}\right)$.}

\tf{F}{The empty set is a subspace of $\mathbb{R}^2$.}

\tf{T}{The set of points on the line defined by $y = 2x$ is a subspace of $\mathbb{R}^2$.}

\tf{F}{The set of points on the line defined by $y = 2x + 1$ is a subspace of $\mathbb{R}^2$.}

\tf{F}{The set of solutions to the equation $x^2 + y^2 = 1$ (a circle) is a subspace of $\mathbb{R}^2$.}

\tf{T}{The set of pairs $\{(x,y): x\geq 0, y\geq 0\}$, where both $x$ and $y$ are nonnegative (i.e., the first quadrant), is a subset of $\mathbb{R}^2$.}

\tf{T}{If $S_1, S_2, S_3$ are subspaces of $\mathbb{R}^4$, then the intersection $S_1\cap S_2 \cap S_3$ is also a subspace of $\mathbb{R}^4$.}

\end{enumerate}

{\noindent\bf\color{red} Some Echelon Forms:}
$$\left(\begin{array}{rr}
1 & 2 \\
3 & 4
\end{array}\right) \longrightarrow \left(\begin{array}{rr}
1 & 0 \\
0 & 1
\end{array}\right)$$
$$\left(\begin{array}{rr}
10 & 20 \\
20 & 10 \\
1 & 2
\end{array}\right) \longrightarrow \left(\begin{array}{rr}
1 & 0 \\
0 & 1 \\
0 & 0
\end{array}\right)$$
$$\left(\begin{array}{rrr}
1 & 2 & 3 \\
4 & 5 & 6 \\
7 & 8 & 9
\end{array}\right) \longrightarrow \left(\begin{array}{rrr}
1 & 0 & -1 \\
0 & 1 & 2 \\
0 & 0 & 0
\end{array}\right)$$
$$\left(\begin{array}{rrr}
1 & 2 & 3 \\
4 & 5 & 6 \\
9 & 8 & 7
\end{array}\right) \longrightarrow \left(\begin{array}{rrr}
1 & 0 & -1 \\
0 & 1 & 2 \\
0 & 0 & 0
\end{array}\right)$$
$$\left(\begin{array}{rrrr}
1 & 2 & 3 & 4 \\
5 & 6 & 7 & 8 \\
9 & 10 & 11 & 12
\end{array}\right) \longrightarrow \left(\begin{array}{rrrr}
1 & 0 & -1 & -2 \\
0 & 1 & 2 & 3 \\
0 & 0 & 0 & 0
\end{array}\right)$$


\end{document}